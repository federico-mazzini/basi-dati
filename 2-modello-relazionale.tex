\documentclass{beamer}

\usetheme{Madrid}
\usecolortheme{default}

\usepackage[italian]{babel}
\usepackage[utf8]{inputenc}
\usepackage[T1]{fontenc}
\usepackage{graphicx}
\usepackage{tikz} 
\usetikzlibrary{positioning, shapes, arrows.meta}

\title{Basi di Dati}
\subtitle{Il Modello Relazionale}
\author{Prof. Federico Mazzini, Prof.ssa Francesca Larotonda}
\institute{I.I.S. Francesco Alberghetti}
\date{A.S. 2025 -- 2026}

\begin{document}

%------------------------------------------------
\begin{frame}
\titlepage
\end{frame}

%------------------------------------------------
\begin{frame}{Storia del Modello Relazionale}
Il modello relazionale è stato introdotto da \textbf{Edgar F. Codd} presso \textbf{IBM} negli anni '70.

\medskip
Ha rivoluzionato il modo di organizzare e gestire i dati, proponendo una struttura semplice e potente basata su tabelle.

\medskip
Oggi i database che si basano sul modello relazionale sono chiamati \textbf{RDBMS} (Relational Database Management Systems). Questa tipologia di database è oggi la più diffusa e utilizzata in quasi tutti i contesti applicativi.
\end{frame}

%------------------------------------------------
\begin{frame}{Il modello relazionale}

\begin{block}{Definizione}
Il modello relazionale organizza i dati in tabelle chiamate \textbf{relazioni}, dove ogni tabella rappresenta un'entità o un concetto del mondo reale, e ogni riga (tupla) rappresenta un'istanza specifica di quell'entità.
\end{block}

\medskip
\begin{tabular}{|c|c|c|c|}
\hline
Cliente & Nome & Cognome & Email \\
\hline
1 & Mario & Rossi & mario.rossi@email.com \\
2 & Anna & Bianchi & anna.bianchi@email.com \\
3 & Luca & Verdi & luca.verdi@email.com \\
\hline
\end{tabular}

\end{frame}

%------------------------------------------------
\begin{frame}{Concetti chiave del Modello Relazionale}

\medskip
\begin{tabular}{|c|c|c|c|}
\hline
Cliente & Nome & Cognome & Email \\
\hline
1 & Mario & Rossi & mario.rossi@email.com \\
2 & Anna & Bianchi & anna.bianchi@email.com \\
3 & Luca & Verdi & luca.verdi@email.com \\
\hline
\end{tabular}
\medskip
\begin{itemize}
    \item La tabella rappresenta una \textbf{relazione} chiamata "Clienti".
    \item Le colonne (Cliente, Nome, Cognome, Email) sono gli \textbf{attributi} della relazione.
    \item Le righe sono le \textbf{istanze} della relazione, che rappresentano i singoli clienti con i loro dati specifici.
    \item L'intestazione della tabella (Nome della tabella + Nome attributi) costituisce lo \textbf{schema} della relazione.
\end{itemize}

\bigskip
\textbf{Schema relazione:} \texttt{Clienti(IDCliente, Nome, Cognome, Email)}

\end{frame}


%------------------------------------------------
\begin{frame}{Vincoli sull'ordine dei dati}
\begin{tabular}{|c|c|c|c|}
\hline
Cliente & Nome & Cognome & Email \\
\hline
1 & Mario & Rossi & mario.rossi@email.com \\
2 & Anna & Bianchi & anna.bianchi@email.com \\
3 & Luca & Verdi & luca.verdi@email.com \\
\hline
\end{tabular}
\medskip

\textbf{L'ordine dei dati non è rilevante}:

\begin{itemize}
    \item L'ordinamento delle righe non ha importanza: le tuple possono essere elencate in qualsiasi ordine.
    \item L'ordinamento delle colonne non ha importanza: gli attributi possono essere visualizzati in qualsiasi ordine.
\end{itemize}

\medskip
Una relazione è definita dal \textbf{contenuto delle tuple} e dagli \textbf{attributi}, non dalla loro posizione nella tabella.


\end{frame}

%------------------------------------------------
\begin{frame}{Vincoli sulla relazione}
\begin{tabular}{|c|c|c|c|}
\hline
Cliente & Nome & Cognome & Email \\
\hline
1 & Mario & Rossi & mario.rossi@email.com \\
2 & Anna & Bianchi & anna.bianchi@email.com \\
3 & Luca & Verdi & luca.verdi@email.com \\
\hline
\end{tabular}
\medskip

\textbf{La relazione deve rispettare alcuni vincoli}:
\begin{itemize}
    \item \textbf{Ogni attributo deve essere unico:} non possono esistere due colonne con lo stesso nome all'interno della stessa relazione.
    \item \textbf{Ogni tupla deve essere unica:} non possono esistere due righe identiche all'interno della relazione.
    \item \textbf{Omogeneità dei dati per colonna:} tutti i valori di una colonna devono appartenere allo stesso tipo di dato.  
    \textbf{Esempio:} nella colonna \texttt{Cliente} devono esserci solo numeri interi, mentre nella colonna \texttt{Nome} solo stringhe di testo.
\end{itemize}
\alert{Esercizi 1-4 dispense}
\end{frame}

\subsection{Vincoli d'integrità intra-relazionali}
%==============================================================================

\begin{frame}{Vincoli d'integrità intra-relazionali}
\begin{block}{Definizione} Regole opzionali e decise da chi progetta la relazione che garantiscono la correttezza e la coerenza dei dati all'interno di una singola relazione.
\end{block}
\begin{itemize}
    \item Garantiscono coerenza e validità dei dati all'interno di una relazione
    \item Si applicano \textbf{ad ogni singola tupla}
    \item Tipi principali:
    \begin{itemize}
        \item \textbf{Vincoli di dominio}: intervalli, valori ammessi, condizioni logiche
        \item \textbf{Vincoli di non nullità}: alcuni attributi devono sempre avere un valore
    \end{itemize}
\end{itemize}
\end{frame}

%==============================================================================
\begin{frame}{Vincoli di dominio}
\begin{block}{Definizione} Intervalli, valori ammessi e condizioni logiche sulle istanze degli attributi
\end{block}

\textbf{Relazione:} \texttt{Voto(Matricola, CodiceMateria, Voto)}

\begin{center}
\begin{tabular}{|c|c|c|}
\hline
\textbf{Matricola} & \textbf{CodiceMateria} & \textbf{Voto} \\
\hline
101 & MAT01 & 8 \\
102 & INF02 & 11 \\
103 & FIS03 & 9 \\
\hline
\end{tabular}
\end{center}

\begin{itemize}
    \item Il voto deve appartenere all’intervallo \([0,10]\)
    \item Il valore \texttt{11} viola il vincolo di dominio
\end{itemize}
\end{frame}

%==============================================================================
\begin{frame}{Vincoli di non nullità}

\begin{block}{Definizione} Alcuni attributi devono sempre avere un valore e non possono essere vuoti
\end{block}

\textbf{Relazione:} \texttt{Studente(Matricola, Nome, Cognome)}

\begin{center}
\begin{tabular}{|c|c|c|}
\hline
\textbf{Matricola} & \textbf{Nome} & \textbf{Cognome} \\
\hline
101 & Marco & Rossi \\
102 &  & Bianchi \\
103 & Luca & Verdi \\
\hline
\end{tabular}
\end{center}

\begin{itemize}
    \item La seconda tupla viola il vincolo di non nullità
    \item L’attributo \texttt{Nome} deve essere sempre valorizzato
\end{itemize}
\end{frame}

%==============================================================================
\begin{frame}{Vincoli di dominio -- esempi avanzati}
\begin{itemize}
    \item \textbf{Relazione Dipendente(Matricola, Nome, Cognome, Stipendio, Età)}
    \begin{itemize}
        \item \texttt{Stipendio} $\ge$ 0
        \item \texttt{Età} $\in [18,70]$
    \end{itemize}
    \item \textbf{Relazione Prodotto(Codice, Nome, Prezzo, Quantità)}
    \begin{itemize}
        \item \texttt{Prezzo} $\ge$ 0
        \item \texttt{Quantità} $\in \mathbb{N}$ (numeri interi non negativi)
    \end{itemize}
\end{itemize}

\begin{center}
\begin{tabular}{|c|c|c|c|}
\hline
Matricola & Nome & Stipendio & Età \\
\hline
101 & Marco & 2000 & 30 \\
102 & Anna & -1500 & 25 \\
103 & Luca & 1800 & 72 \\
\hline
\end{tabular}
\end{center}

\begin{itemize}
    \item Tupla 102 viola il vincolo di dominio (stipendio negativo)
    \item Tupla 103 viola il vincolo di dominio (età > 70)
\end{itemize}

\alert{Esercizi 5-8 dispense}
\end{frame}


%==============================================================================
\begin{frame}{Vincoli di chiave (Primary Key)}

\begin{block}{Definizione} Per \textbf{chiave di una relazione} si intende un insieme di attributi che consente di identificare in maniera univoca le ennuple di una relazione.
\end{block}
    
\begin{itemize}
    \item Identificare in modo univoco le tuple di una relazione
    \item Concetti fondamentali:
    \begin{itemize}
        \item \textbf{Superchiave:} insieme di attributi che identifica univocamente le tuple
        \item \textbf{Superchiave minimale:} insieme minimo di attributi che identifica univocamente le tuple di una relazione non contiene attributi ridondanti
    \end{itemize}
\end{itemize}
\end{frame}

\begin{frame}{Esempio 1 -- Relazione Prodotti}
Relazione \texttt{Prodotto(CodiceProdotto, NomeProdotto, Categoria, Prezzo)}

\begin{center}
\begin{tabular}{|c|c|c|c|}
\hline
\textbf{CodiceProdotto} & \textbf{NomeProdotto} & \textbf{Categoria} & \textbf{Prezzo} \\
\hline
P001 & Smartphone X & Telefonia & 599.00 \\
P002 & Laptop Pro & Informatica & 1200.00 \\
P003 & Tablet Y & Informatica & 399.00 \\
\hline
\end{tabular}
\end{center}

\begin{itemize}
    \item \{\texttt{CodiceProdotto}\} è una \textbf{superchiave minima}.
    \item \{\texttt{CodiceProdotto, NomeProdotto}\} è una \textbf{superchiave}, ma non minima.
    \item \{\texttt{CodiceProdotto, NomeProdotto, Prezzo}\} è ancora una superchiave con attributi ridondanti.
\end{itemize}
\end{frame}


%==============================================================================
\begin{frame}{Esempio 2 -- Biblioteca}
\textbf{Relazione:} \texttt{Libro(ISBN, Titolo, Autore, AnnoPubblicazione)}

\begin{center}
\begin{tabular}{|c|c|c|c|}
\hline
\textbf{ISBN} & \textbf{Titolo} & \textbf{Autore} & \textbf{Anno} \\
\hline
978-88-04-51234-5 & Il nome della rosa & Umberto Eco & 1980 \\
978-88-04-51235-2 & Il fu Mattia Pascal & Luigi Pirandello & 1904 \\
978-88-04-51236-9 & Se questo è un uomo & Primo Levi & 1947 \\
978-88-04-51237-6 & Il barone rampante & Italo Calvino & 1957 \\
\hline
\end{tabular}
\end{center}

\begin{itemize}
    \item \{\texttt{ISBN}\} è una superchiave minimale $\Rightarrow$ chiave
    \item \{\texttt{ISBN, Titolo}\} è una superchiave ma non minima
    \item \{\texttt{Titolo, Autore, Anno}\} può essere superchiave solo se ogni libro ha titolo unico nello stesso anno
\end{itemize}
\end{frame}


%==============================================================================
\begin{frame}{Esempio 3 -- Acquisti di un cliente}
\textbf{Relazione:} \texttt{Ordine(NumeroOrdine, Data, Cliente, Importo)}

\begin{center}
\begin{tabular}{|c|c|c|c|}
\hline
\textbf{Data} & \textbf{Cliente} & \textbf{Importo} \\
\hline
2024-06-01 & Rossi & 120.00 \\
2024-06-02 & Bianchi & 85.50 \\
2024-06-02 & Neri & 42.00 \\
2024-06-03 & Rossi & 75.00 \\
\hline
\end{tabular}
\end{center}

Quale superchiave minima scegliere?

\end{frame}

%==============================================================================
\begin{frame}{Esempio 3 -- Acquisti di un cliente}
\textbf{Relazione:} \texttt{Ordine(NumeroOrdine, Data, Cliente, Importo)}

\begin{center}
\begin{tabular}{|c|c|c|c|}
\hline
\textbf{Data} & \textbf{Cliente} & \textbf{Importo} \\
\hline
2024-06-01 & Rossi & 120.00 \\
2024-06-02 & Bianchi & 85.50 \\
2024-06-02 & Neri & 42.00 \\
2024-06-03 & Rossi & 75.00 \\
\hline
\end{tabular}
\end{center}

Quale superchiave minima scegliere?
\alert{E se un cliente compra lo stesso giorno più volte?}
\end{frame}


%==============================================================================
\begin{frame}{Esempio 3 -- Acquisti di un cliente}
\textbf{Relazione:} \texttt{Ordine(NumeroOrdine, Data, Cliente, Importo)}

\begin{center}
\begin{tabular}{|c|c|c|c|}
\hline
\textbf{Data} & \textbf{Cliente} & \textbf{Importo} \\
\hline
2024-06-01 & Rossi & 120.00 \\
2024-06-02 & Bianchi & 85.50 \\
2024-06-02 & Neri & 42.00 \\
2024-06-03 & Rossi & 75.00 \\
\hline
\end{tabular}
\end{center}

Quale superchiave minima scegliere?
\alert{E se un cliente compra lo stesso giorno più volte gli stessi prodotti?}
\end{frame}

%==============================================================================
\begin{frame}{Esempio 3 -- Acquisti di un cliente}
\textbf{Relazione:} \texttt{Ordine(NumeroOrdine, Data, Cliente, Importo)}

\begin{center}
\begin{tabular}{|c|c|c|c|}
\hline
\textbf{NumeroOrdine} & \textbf{Data} & \textbf{Cliente} & \textbf{Importo} \\
\hline
5001 & 2024-06-01 & Rossi & 120.00 \\
5002 & 2024-06-02 & Bianchi & 85.50 \\
5003 & 2024-06-02 & Neri & 42.00 \\
5004 & 2024-06-03 & Rossi & 75.00 \\
\hline
\end{tabular}
\end{center}

\alert{Aggiungiamo un attributo univoco!}

\begin{itemize}
    \item \{\texttt{NumeroOrdine}\} è superchiave minimale e chiave primaria
\end{itemize}
\end{frame} 

%==============================================================================
\begin{frame}{Chiave primaria}
Una relazione può avere \textbf{più superchiavi minimali}, è necessario sceglierne \textbf{una sola} come riferimento principale. 
\medskip

\begin{block}{Definizione}
La chiave primaria di una relazione è una \textbf{superchiave minimale scelta} come identificatore ufficiale della relazione;
\end{block}
\medskip
\begin{itemize}
    \item identifica in modo univoco ogni tupla;
    \item permette al DBMS di gestire correttamente i dati;
    \item viene utilizzata da altre relazioni per creare collegamenti.
\end{itemize}

\medskip
\alert{La differenza non è teorica, ma progettuale: la chiave primaria è la superchiave minimale che decidiamo di usare davvero.}
\alert{Esercizi 9 - 13 dispense}
\end{frame}

\end{document}