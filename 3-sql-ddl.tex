\documentclass{beamer}

\usetheme{Madrid}
\usecolortheme{default}

\usepackage[italian]{babel}
\usepackage[utf8]{inputenc}
\usepackage[T1]{fontenc}
\usepackage{xcolor} 
\usepackage{colortbl}
\usepackage{verbatim}

\title{Basi di Dati}
\subtitle{SQL DDL -- Definizione delle strutture}
\author{Prof. Federico Mazzini, Prof.ssa Francesca Larotonda}
\institute{I.I.S. Francesco Alberghetti}
\date{A.S. 2025 -- 2026}

\begin{document}

%==============================================================================
\begin{frame}
\titlepage
\end{frame}

%==============================================================================
\section{SQL DDL (Data Definition Language)}
%==============================================================================

\begin{frame}{Dal modello al database}
Fino ad ora abbiamo:
\begin{itemize}
    \item analizzato il mondo reale;
    \item costruito uno schema concettuale;
    \item definito uno schema relazionale;
    \item individuato chiavi e vincoli.
\end{itemize}

\medskip
\alert{Ora dobbiamo comunicare tutto questo al DBMS.}
\end{frame}

%==============================================================================
\begin{frame}{Come comunichiamo con il DBMS}
Il DBMS:
\begin{itemize}
    \item non interpreta diagrammi;
    \item non capisce schemi su carta;
    \item non intuisce i vincoli logici.
\end{itemize}

\medskip
\alert{Il DBMS capisce solo istruzioni formali.}

\medskip
Serve un linguaggio standard.
\end{frame}

%==============================================================================
\begin{frame}{SQL}
\begin{block}{Definizione}
SQL (Structured Query Language) è il linguaggio standard per comunicare con i database relazionali.
\end{block}

\begin{itemize}
    \item è un linguaggio \textbf{dichiarativo};
    \item descrive \textbf{cosa} vogliamo, non \textbf{come};
    \item è utilizzato da tutti i principali DBMS.
\end{itemize}
\end{frame}

%==============================================================================
\begin{frame}{Le categorie di SQL}
SQL è suddiviso in più sotto-linguaggi:
\begin{itemize}
    \item \textbf{DDL} -- Data Definition Language;
    \item DML -- Data Manipulation Language;
    \item DQL -- Data Query Language;
    \item DCL -- Data Control Language.
\end{itemize}

\medskip
In questo modulo ci concentriamo sul:
\begin{center}
\alert{\textbf{DDL}}
\end{center}
\end{frame}

%==============================================================================
\begin{frame}{DDL}
\begin{block}{Definizione}
Il DDL è la parte di SQL che serve per \textbf{definire e modificare la struttura del database}.
\end{block}

Con il DDL possiamo:
\begin{itemize}
    \item creare tabelle e database;
    \item modificare la struttura;
    \item eliminare oggetti;
    \item definire vincoli di integrità.
\end{itemize}

\medskip
\alert{Il DDL traduce il modello relazionale in istruzioni per il DBMS.}
\end{frame}

%==============================================================================
\begin{frame}{DDL e DBMS}
SQL è uno standard, ma ogni DBMS ha delle varianti.

\medskip
In questo corso utilizziamo:
\begin{center}
\textbf{MariaDB}
\end{center}

\begin{itemize}
    \item DBMS compatibile MySQL;
    \item usato tramite phpMyAdmin;
    \item con sintassi e tipi di dato specifici.
\end{itemize}

\alert{Gli esempi mostrati sono validi per MariaDB.}
\end{frame}

%==============================================================================
\subsection{CREATE}
%==============================================================================

\begin{frame}{CREATE}
Il comando \textbf{CREATE} serve per creare nuovi oggetti nel database:
\begin{itemize}
    \item tabelle;
    \item database;
    \item viste;
    \item indici.
\end{itemize}
\end{frame}

%==============================================================================
\begin{frame}{CREATE TABLE}
\begin{block}{Sintassi generale}
\small
\texttt{
CREATE TABLE nome\_tabella ( \\
\hspace{0.5cm} colonna tipo\_dato [vincoli], \\
\hspace{0.5cm} ... \\
);
}
\end{block} 

\medskip
\alert{Il DBMS non intuisce nulla: tutto va dichiarato.}
\end{frame}

%==============================================================================
\begin{frame}[fragile]{Esempio CREATE TABLE}
\small
\begin{verbatim}
CREATE TABLE Studente (
    Matricola INT PRIMARY KEY,
    Nome VARCHAR(50),
    Cognome VARCHAR(50),
    DataNascita DATE
);
\end{verbatim}

\medskip
\begin{itemize}
    \item \texttt{Matricola} è chiave primaria;
    \item \texttt{Nome} e \texttt{Cognome} sono stringhe;
    \item \texttt{DataNascita} è un campo data.
\end{itemize}
\end{frame}

%==============================================================================
\subsection{Tipi di dato in MariaDB}
%==============================================================================

\begin{frame}{Tipi di dato}
Ogni colonna deve avere un tipo di dato che definisce:
\begin{itemize}
    \item i valori ammessi;
    \item lo spazio occupato;
    \item le operazioni possibili.
\end{itemize}
\end{frame}

%==============================================================================
\begin{frame}{Tipi numerici}
\begin{itemize}
    \item \texttt{INT} -- numeri interi;
    \item \texttt{SMALLINT}, \texttt{BIGINT};
    \item \texttt{DECIMAL(p,s)} -- numeri decimali precisi.
\end{itemize}
\end{frame}

%==============================================================================
\begin{frame}{Tipi testuali}
\begin{itemize}
    \item \texttt{CHAR(n)} -- lunghezza fissa;
    \item \texttt{VARCHAR(n)} -- lunghezza variabile;
    \item \texttt{TEXT} -- testi lunghi.
\end{itemize}
\end{frame}

%==============================================================================
\begin{frame}{Tipi temporali}
\begin{itemize}
    \item \texttt{DATE};
    \item \texttt{TIME};
    \item \texttt{DATETIME}.
\end{itemize}
\end{frame}

%==============================================================================
\begin{frame}{Scelta dei tipi}
\begin{itemize}
    \item CodiceFiscale $\rightarrow$ \texttt{CHAR(16)};
    \item Nome, Cognome $\rightarrow$ \texttt{VARCHAR};
    \item Prezzo $\rightarrow$ \texttt{DECIMAL}.
\end{itemize}

\medskip
\alert{Il tipo di dato riflette il significato dell’informazione.}
\end{frame}

%==============================================================================
\subsection{Vincoli}
%==============================================================================

\begin{frame}{Vincoli (Constraints)}
I vincoli garantiscono l’integrità dei dati.

\begin{itemize}
    \item \texttt{PRIMARY KEY};
    \item \texttt{FOREIGN KEY};
    \item \texttt{UNIQUE};
    \item \texttt{NOT NULL};
    \item \texttt{CHECK}.
\end{itemize}
\end{frame}

%==============================================================================
\begin{frame}[fragile]{PRIMARY KEY}
\small
\begin{verbatim}
CREATE TABLE Studente (
    Matricola INT PRIMARY KEY,
    Nome VARCHAR(50),
    Cognome VARCHAR(50)
);
\end{verbatim}

\medskip
\begin{itemize}
    \item identifica univocamente ogni riga;
    \item non può essere NULL.
\end{itemize}
\end{frame}

%==============================================================================
\begin{frame}[fragile]{AUTO\_INCREMENT in MariaDB}
\small
\begin{verbatim}
CREATE TABLE Studente (
    Matricola INT AUTO_INCREMENT,
    Nome VARCHAR(50) NOT NULL,
    Cognome VARCHAR(50) NOT NULL,
    PRIMARY KEY (Matricola)
);
\end{verbatim}

\medskip
\alert{Il DBMS genera automaticamente la chiave primaria.}
\end{frame}

%==============================================================================
\begin{frame}[fragile]{FOREIGN KEY}
\small
\begin{verbatim}
FOREIGN KEY (Matricola)
REFERENCES Studente(Matricola)
\end{verbatim}

\medskip
\begin{itemize}
    \item garantisce la referenzialità;
    \item impedisce riferimenti a dati inesistenti.
\end{itemize}
\end{frame}

%==============================================================================
\begin{frame}[fragile]{NOT NULL e UNIQUE}
\small
\begin{verbatim}
Nome VARCHAR(50) NOT NULL,
Email VARCHAR(100) UNIQUE
\end{verbatim}

\medskip 
\begin{itemize}
    \item \texttt{NOT NULL}: valore obbligatorio;
    \item \texttt{UNIQUE}: valori non ripetuti.
\end{itemize}
\end{frame}

%==============================================================================
\begin{frame}[fragile]{CHECK in MariaDB}
\small
\begin{verbatim}
CREATE TABLE Studente (
    Eta INT CHECK (Eta >= 18)
);
\end{verbatim}

\medskip
\alert{Nelle versioni moderne di MariaDB il vincolo viene applicato.}
\end{frame}

%==============================================================================
\subsection{DROP e ALTER}
%==============================================================================

\begin{frame}[fragile]{DROP}
\small
\begin{verbatim}
DROP TABLE Studente;
\end{verbatim}

\medskip
\alert{Comando irreversibile: struttura e dati vengono eliminati.}
\end{frame}

%==============================================================================
\begin{frame}[fragile]{ALTER TABLE}
\small
\begin{verbatim}
ALTER TABLE Studente ADD Email VARCHAR(100);
ALTER TABLE Studente MODIFY Cognome VARCHAR(100);
ALTER TABLE Studente DROP COLUMN Email;
\end{verbatim}
\end{frame}

%==============================================================================
\begin{frame}[fragile]{ALTER TABLE e dati esistenti}
\begin{itemize}
    \item la tabella può contenere dati;
    \item i vincoli vengono controllati;
    \item alcune modifiche possono fallire.
\end{itemize}

\alert{DDL lavora sulla struttura, ma può impattare sui dati.}
\end{frame}

%==============================================================================
\section{Conclusione}
%==============================================================================

\begin{frame}{DDL e modello relazionale}
\begin{itemize}
    \item Entità $\rightarrow$ Tabelle;
    \item Attributi $\rightarrow$ Colonne;
    \item Chiavi $\rightarrow$ Vincoli;
    \item Relazioni $\rightarrow$ Foreign key.
\end{itemize}

\medskip
\alert{DDL è il ponte tra teoria e database reale.}
\end{frame}

\end{document}
