\documentclass{beamer}

\usetheme{Madrid}
\usecolortheme{default}

\usepackage[italian]{babel}
\usepackage[utf8]{inputenc}
\usepackage[T1]{fontenc}
\usepackage{graphicx}
\usepackage{tikz}
\usepackage{booktabs}
\usetikzlibrary{positioning, shapes, arrows.meta} 

\title{Basi di Dati}
\subtitle{Introduzione ai dati e i DBMS}
\author{Prof. Federico Mazzini}
\institute{I.I.S. Francesco Alberghetti}
\date{A.S. 2025 -- 2026}

\begin{document}

%------------------------------------------------
\begin{frame}
\titlepage
\end{frame}

%------------------------------------------------
\begin{frame}{Introduzione}
Ogni applicazione è composta principalmente da due parti principali:
\begin{itemize}
    \item parte algoritmica
    \item parte di memorizzazione e gestione dati
\end{itemize}

\medskip
\alert{In questo unità ci concentreremo sulla parte di gestione dei dati}
\end{frame}

%------------------------------------------------
\begin{frame}{I dati nelle applicazioni di tutti i giorni}
Ogni applicazione che usiamo gestisce dati: dai social network al registro elettronico, tutto ruota intorno ai dati.
\begin{itemize}
    \item \textbf{Registro elettronico}: voti, assenze, compiti e comunicazioni sono dati che vengono registrati, aggiornati e consultati ogni giorno da studenti, insegnanti e famiglie.
    \item \textbf{Social network}: ogni post, commento, like, amicizia è un dato che viene salvato, collegato agli utenti e reso disponibile in tempo reale.
    \item \textbf{Sistemi aziendali}: ordini, clienti, fatture, prodotti... in azienda tutto viene gestito tramite dati memorizzati in modo strutturato.
    \item \textbf{Vita quotidiana}: anche la rubrica del telefono, la cronologia delle ricerche, le playlist musicali sono dati che un'applicazione gestisce per noi.
\end{itemize}
\end{frame}

%------------------------------------------------
\begin{frame}{Dato e Informazione: definizioni}
\begin{block}{Dato}
Rappresentazione oggettiva e grezza di un fatto, senza contesto
\end{block}

\begin{block}{Informazione}
Dato interpretato e contestualizzato per assumere significato
\end{block}

\medskip
\alert{La trasformazione da dato a informazione richiede contesto e interpretazione}
\end{frame}

%------------------------------------------------
\begin{frame}{Esempi di dato e informazione}
\begin{itemize}
    \item \texttt{8} → dato numerico semplice
    \item \texttt{8 = voto di Informatica di uno studente} → informazione contestualizzata
    \item \texttt{"Mario Rossi ha 8 in Informatica"} → informazione completa e significativa
\end{itemize}

\medskip
\textbf{Il significato di un dato nasce dal contesto e dalle relazioni tra dati}
\medskip
\alert{Gestire i dati in modo organizzato è fondamentale per trasformarli in informazioni utili}
\end{frame}

%------------------------------------------------
\begin{frame}{Sistema Informativo}
\begin{block}{Definizione}
Insieme di informazioni, procedure e persone che supportano le attività di un'organizzazione
\end{block}

\medskip
\alert{L’esistenza di un Sistema Informativo è indipendente dalla sua automatizzazione.}
\end{frame}


%------------------------------------------------
\begin{frame}{Sistema Informativo}
\begin{block}{Definizione}
Insieme di informazioni, procedure e persone che supportano le attività di un'organizzazione
\end{block}

\medskip
\alert{L’esistenza di un Sistema Informativo è indipendente dalla sua automatizzazione.}

{\tiny Nell’Antica Roma venivano già raccolti dati sulle persone attraverso i censimenti.
I cittadini venivano registrati in elenchi in base all’età, alla ricchezza e ai diritti civili.
Questi dati servivano per decidere chi poteva votare, chi doveva pagare le tasse e chi doveva andare nell’esercito.}
\end{frame}

%------------------------------------------------
\begin{frame}{Sistema Informatico}
Il ruolo dell'informatica ha radicalmente cambiato i sistemi informativi. Un sistema informatico è quindi una sottoparte del sistema informativo che permette di:
\begin{itemize}
\item automatizzare i processi
\item gestire i dati (e le informazioni) in maniera più efficiente ed efficace.  
\end{itemize}

\medskip
{\tiny Oggi un sistema informativo aziendale gestisce ad esempio tramite interfaccia web gli acquisti di un'azienda, registrando automaticamente ordini e producendo fatture. I dati di questi processi vengono memorizzati all'interno di dispositivi informatici e resi disponibili ai dipendenti autorizzati in qualsiasi momento.}
\end{frame}

%------------------------------------------------
\begin{frame}{Approcci di gestione dei dati}
    Gran parte dei sistemi informatici ha necessità di gestire i dati in maniera persistente. Per farlo sono possibili due approcci principali:
\begin{itemize}
    \item Approccio basato su File System (software non specializzati)
    \item Approccio basato su DBMS (software specializzato)
\end{itemize}

\medskip
\alert{Le basi di dati sono il cuore della gestione delle informazioni}
\end{frame}

%------------------------------------------------
\begin{frame}{Approccio basato su File System}
\centering
\includegraphics[width=0.5\textwidth]{images/software-spec.png}
\begin{itemize}
    \item L'applicazione contiene sia la logica applicativa e algoritmica, che quella di gestione dei dati
    \item Ogni software è scritto da persone diverse con regole e approcci diversi
    \item Nessuna condivisione centralizzata
    \item Dati duplicati e incoerenti
\end{itemize}

\medskip
\end{frame}

%------------------------------------------------
\begin{frame}{Approccio DBMS}
\begin{block}{DBMS}
Un DBMS è un sistema software che è in grado di gestire collezioni di dati grandi, condivise e persistenti, in maniera efficiente e sicura.
\end{block}


\centering
\includegraphics[width=0.5\textwidth]{images/dbms-app.jpg}

\medskip
\alert{Oggi, qualsiasi applicazione che gestisce dati utilizza un DBMS}
\end{frame}

%------------------------------------------------
\begin{frame}{Caratteristiche base DBMS}

\begin{itemize}
    \item \textbf{Memorizza dati}: tutto organizzato su disco, non sparso in file casuali
    \item \textbf{Legge e scrive dati}: aggiungi, modifichi, cancelli senza pensare al “come”
    \item \textbf{Condivisione}: più utenti e applicazioni possono usare gli stessi dati
    \item \textbf{Sicurezza}: solo chi ha permesso può leggere o scrivere
    \item \textbf{Affidabilità}: in caso di guasti, niente dati persi
    \item \textbf{Integrità dei dati}: regole rispettate (es. voto massimo 10)
    \item \textbf{Interrogazione veloce}: trova info senza cercare manualmente
    \item \textbf{Gestione concorrenza}: più utenti contemporaneamente senza conflitti
\end{itemize}

\end{frame}

%------------------------------------------------
\begin{frame}{DBMS come gestore centrale}
\centering
\includegraphics[width=0.5\textwidth]{images/integrated-database-DBMS.png}

\end{frame}

%------------------------------------------------
\begin{frame}{Perché un DBMS è fondamentale}
\textbf{Gestione di grandi quantità di dati}

\medskip
\begin{itemize}
    \item Amazon Prime: oltre 240 milioni di abbonati.
    \item Facebook: più di 3 miliardi di utenti attivi mensili.
    \item Instagram: oltre 1,3 miliardi di immagini caricate ogni giorno.
\end{itemize}


\end{frame}


%------------------------------------------------
\begin{frame}{Perché un DBMS è fondamentale}
\textbf{Gestione della concorrenza}: il DBMS coordina l'accesso simultaneo ai dati.

\medskip
\textbf{Scenario:} due utenti prelevano contemporaneamente dal conto X (saldo iniziale 120 €).  

\medskip
\begin{tabular}{c c c}
\toprule
Tempo & Operazione & Valore di X \\
\midrule
$t_1$ & OP1: Leggi saldo & 120 \\
$t_2$ & OP2: Leggi saldo & 120 \\
$t_3$ & OP1: Calcola saldo-100 & 120 \\
$t_4$ & OP2: Calcola saldo-80 & 120 \\
$t_5$ & OP1: Scrivi saldo finale & 20 \\
$t_6$ & OP2: Scrivi saldo finale & 40 \\
\bottomrule
\end{tabular}

\medskip
\alert{Saldo finale = 40 €}

\medskip
\textbf{Soluzione:} il DBMS serializza le operazioni, così il risultato finale corrisponde a eseguire le operazioni una alla volta.
\end{frame}


%------------------------------------------------
\begin{frame}{Perché un DBMS è fondamentale}
\textbf{Affidabilità}: il DBMS garantisce protezione da malfunzionamenti e guasti.

\medskip
\textbf{Scenario:} trasferimento 200 € da conto A (saldo 500 €) a conto B (saldo 300 €).  

\medskip
Operazioni:
\begin{itemize}
    \item Addebitare 200 € dal conto A
    \item Accreditare 200 € sul conto B
\end{itemize}

\medskip
\textbf{Problema:} se il sistema si interrompe dopo l’addebito, ma prima dell’accredito → dati incoerenti.  

\medskip
\textbf{Soluzione DBMS:} uso delle \textbf{transazioni}:
\begin{itemize}
    \item Entrambe le operazioni completate correttamente, oppure
    \item Nessuna delle due operazioni viene effettuata.
\end{itemize}
\end{frame}


\begin{frame}{DBMS (Database Management System) come tecnologia}

\centering
\includegraphics[width=1\textwidth]{images/database-landscape.jpeg}

\end{frame}



%------------------------------------------------
\begin{frame}{DBMS come astrazione}
\textbf{Un DBMS permette agli utenti e alle applicazioni di lavorare con i dati senza dover conoscere i dettagli di memorizzazione:}

\medskip
\begin{itemize}
    \item \textbf{Come} i dati sono memorizzati e gestiti è nascosto
    \item \textbf{Cosa} voglio ottenere (dati, informazioni)
    \item Il DBMS gestisce automaticamente tutti i dettagli tecnici.
    \item L’utente interagisce con un livello alto di astrazione
    \begin{itemize}
    \item "Dammi tutti i clienti che hanno speso più di 100 €”
    \end{itemize}
\end{itemize}

\medskip
\alert{Astrazione = nascondere dettagli implementativi complessi per semplificare l'interazione con il sistema}
\end{frame}


%------------------------------------------------
\begin{frame}{DBMS come astrazione}
\centering
\includegraphics[width=0.5\textwidth]{images/three-levels-database-architecture.jpg}
\end{frame}
\end{document}